% 
% Annual Cognitive Science Conference
% Sample LaTeX Paper -- Proceedings Format
% 

%% Change "letterpaper" in the following line to "a4paper"e if you must.

\documentclass[10pt,letterpaper]{article}

\usepackage{cogsci}
% Recommended, but optional, packages for figures and better typesetting:
\usepackage[margin=1in]{geometry} 
\usepackage{microtype}
\usepackage{graphicx}
\usepackage{subfigure}
\usepackage{booktabs} % for professional tables
\usepackage{nicefrac}       % compact symbols for 1/2f, etc.
\usepackage{microtype}      % microtypography
\usepackage{float}
\usepackage[colorlinks,allcolors=purple]{hyperref}
\usepackage{algorithm}
\usepackage{amsmath}
\usepackage{graphicx}
\usepackage[table,xcdraw]{xcolor}
\usepackage{gensymb}
\usepackage{stmaryrd}
\usepackage{amssymb}
\usepackage{todonotes}
\usepackage{comment}

\cogscifinalcopy % Uncomment this line for the final submission 
\newcommand{\jda}[1]{{\color{blue}[jda: #1]}}


\usepackage{pslatex}
\usepackage{apacite}


%\usepackage[none]{hyphenat} % Sometimes it can be useful to turn off
%hyphenation for purposes such as spell checking of the resulting
%PDF.  Uncomment this block to turn off hyphenation.


\setlength\titlebox{4.5cm}
% You can expand the titlebox if you need extra space
% to show all the authors. Please do not make the titlebox
% smaller than 4.5cm (the original size).
%%If you do, we reserve the right to require you to change it back in
%%the camera-ready version, which could interfere with the timely
%%appearance of your paper in the Proceedings.

\title{Supplemental: Identifying concept libraries \\ from language about object structure}
\vskip 0.3in
\begin{document}

\maketitle



\section{S1. Part I Supplemental Details} \label{sec:s1_part_i}

\subsection{Stimulus generation}

\subsubsection{Structures stimulus generation}

We first defined groups of low-level part abstractions: \texttt{tiles}, \texttt{arches}, and \texttt{house-parts} contained fixed arrangements of blocks; \texttt{rows(width)} and \texttt{pillars(height)} were parameterized functions.
Different subsets of these were hierarchically combined in different ways to create each subdomain.
Each subdomain was parameterized by numerical parameters (e.g. number of arches), as well as by part type parameters (e.g. wall tile).

\texttt{Bridges} contained up to 2 external arches, and up to 5 internal arches of a different type. Each low-level \texttt{arch} abstraction contained two red pillars, which could all be extended up to a maximum height. Each bridge could support a mid-level \texttt{viaduct} abstraction consisting of multiple rows. We also defined a \texttt{suspension(suspension-type)} function that placed red blocks directly above the pillars, of a \texttt{uniform} height, or that \texttt{decreased} or \texttt{increased} in height towards the center of the bridge.

\texttt{Cities} contained two skyscrapers placed a random distance apart. Each skyscraper was defined by a wall \texttt{tile}, which was  \texttt{stacked(height)} vertically and optionally \texttt{mirrored}, and topped with a \texttt{row} or \texttt{pyramid} \texttt{roof}. 

\texttt{Houses} were each topped with a \texttt{pyramid} \texttt{roof}, defined by the width of the house. The width of the house was determined by the width of the first \texttt{floor}, which could contained any permutation of \texttt{{window, bricks, door}}. Up to two additional floors contained permutations of \texttt{{window, bricks}}.

\texttt{Castles} were defined by a central mirrored \texttt{wall} of \texttt{tiles}, and two flanking \texttt{stacks} of \texttt{tiles}, each parameterized by a height (with \texttt{stack height < wall height}), and the wall additionally parameterized by width. The wall and stacks were topped with the same type of \texttt{roof}, providing there was space. Each roof could be a \texttt{pyramid} or \texttt{dome}- similar to a pyramid with an additional shorter row beneath.

We exhaustively enumerated items from each subdomain (with a small range for each parameter), rejecting any tower that extended beyond a 20x20 grid.


\subsubsection{Drawings stimulus generation}

\subsubsection{Generation of programs}

\subsection{Language pre-preprocessing}

\subsection{Results: relationship between base-library programs and length of linguistic descriptions} 

\subsection{Results: pointwise mutual information for subdomain words} 


\section{S2. Part II Supplemental Details} \label{sec:s1_part_ii}

\subsection{Defining a hypothesis space over graphics libraries}

\subsection{Library-to-vocabulary alignment model}


\bibliographystyle{apacite}

\setlength{\bibleftmargin}{.125in}
\setlength{\bibindent}{-\bibleftmargin}

\bibliography{CogSci_Template}


\end{document}
